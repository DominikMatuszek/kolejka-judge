\subsection{JAR files}

\begin{minted}[linenos]{python}
checking.add_steps(
    rename=RenameJavaFile('solution.java'),
    list_jars=ListFiles('**/*.jar', variable_name='jars'),
    compile=CompileJava('**/*.java', compilation_options=[
        '-cp',
        DependentExpr('jars', func=lambda x: '.:' + ':'.join(x))
    ]),
    create_jar=CreateJar('**/*.class', entrypoint='Main'),
    run_jar=RunJarSolution(interpreter_options=[
        DependentExpr(
            'jars',
            func=lambda x: '-Xbootclasspath/a:' + ':'.join(x)
        )
    ], stdout='out'),
    diff=Diff(),
)
status_code, detailed_result = checking.run()
\end{minted}

\code{RenameJavaFile} takes the file submitted to the judge -- whose name might have been mangled in the process -- and
renames it to correspond to the class name that it contains.
\code{ListFiles} takes the glob matching all \code{*.jar} files recursively, and stores all of them in the environment's
variable called \code{jars}.
\code{CompileJava} compiles all matching \code{*.java} files, while overriding the classpath with a
\code{DependentExpr}.
This expression is evaluated just before compilation, using a lambda function that takes only one argument -- since
there's only one variable being evaluated, \code{jars} -- and produces an output of the form
\shell{.:<jar_1>:<jar_2>[...]}.
Therefore the executed command will be \shell{javac -cp .:<jar_1>:<jar_2>[...]}.
\code{CreateJar} takes all compiled \code{*.class} files, and creates a JAR file named \shell{archive.jar}
containing a manifest with the entrypoint set to \shell{Main}.
\code{RunJarSolution} runs the JAR archive with the \shell{-Xbootclasspath/a} option generated in a very similar way as
previously, and redirects the output to the \shell{out} file, which is then compared with \shell{wzo} file using the
\code{Diff} class.
This test case assumes the existence of the \shell{solution.java}, \shell{Main.java}, \shell{wzo}, and any supporting
\code{*.jar} files.
