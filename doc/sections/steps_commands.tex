\subsection{Commands}\label{subsec:steps_commands}
    Commands are wrappers for the programs normally used in the pipeline: \shell{g++}, \shell{diff}, etc.
    They take care of making sure that every dependency is properly set up, and reporting the execution status back to
    the \hyperref[sec:environments]{environment}.\\

    Commands' arguments support a special mechanism, called
    \phantomsection \label{sec:DependentExpr} \python{DependentExpr}.
    It represents a lazy object, that takes an unknown number of variable names, and a function which is used to
    reduce the contents stored (via \hyperref[sec:set_variable]{\python{ExecutionEnvironment.set_variable()}} method)
    under those names to a single object.
    For the most common use case, where only one variable name is given and no reduction is needed, the function
    argument can be omitted.
    \python{DependentExpr} will be evaluated only just before the command execution.\\

    The base class of every command step is \python{CommandBase}.

\subsection*{\python{CommandBase}}\label{subsec:CommandBase}

\begin{itemize}[label={}]
    \item \docfunc{__init__(self, limits, user, group)}
        \docfuncdesc{
            Constructor.
            The \python{limits} argument allows the caller to specify the limits for the command execution without
            having to override \hyperref[sec:get_limits]{\python{get_limits()}} method.
            The \python{user} and \python{group} arguments perform the same function, as per
            \hyperref[sec:get_user]{\python{get_user()}} and \hyperref[sec:get_group]{\python{get_group()}}
            methods, respectively.
        }

    \item \phantomsection \label{sec:command_get_env} \docfunc{get_env(self)}
        \docfuncdesc{
            Used to specify step-specific environment variables that should be present during the command execution.
            Values specified here take precedence over those from the
            \hyperref[sec:env_get_env]{\python{ExecutionEnvironment.get_env()}} method.
        }

    \item \phantomsection \label{sec:get_limits} \docfunc{get_limits(self)}
        \docfuncdesc{
            Defines the dictionary containing the limits active during the command execution.
            It will by called by the
            \hyperref[sec:run_command_step]{\python{ExecutionEnvironment.run_command_step()}} method.
            The default implementation simply returns the limits passed to the constructor.
        }

    \item \docfunc{get_command(self)}
        \docfuncdesc{
            Abstract method.
            Returns a list consisting of an executable and its arguments that will be run.
        }

    \item \docfunc{get_stdin_file(self)}
        \docfuncdesc{
            Returns a path to the file that will be redirected as the \shell{stdin} stream to the program.
            Default: \python{None}.
        }

    \item \docfunc{get_stdout_file(self)}
        \docfuncdesc{
            Returns a path to the file to which the \shell{stdout} of the program will be redirected to.
            Default: \code{logs/<step_name>_stdout.txt}.
        }

    \item \docfunc{get_stderr_file(self)}
        \docfuncdesc{
            Returns a path to the file to which the \shell{stderr} of the program will be redirected to.
            Default: \code{logs/<step_name>_stderr.txt}.
        }

    \item \phantomsection \label{sec:postconditions} \docfunc{postconditions(self)}
        \docfuncdesc{
            Returns a list of validators, consisting of the 2-tuples: \python{(validator, exit_status)}.
            Each of the validators will be called after the command execution.
            Refer to the \hyperref[sec:postconditions_desc]{postconditions description} for more information about their
            interaction.\\
            Default: \python{[]}
        }

    \item \docfunc{verify_postconditions(self, result)}
        \docfuncdesc{
            Iterates over the list returned from the \hyperref[sec:postconditions]{\python{postconditions()}} method,
            calls each one of them with the execution statistics as an argument, and returns the appropriate
            exit status if they return \python{False}.
        }

    \item \phantomsection \label{sec:prerequisites_command} \docfunc{prerequisites(self)}
        \docfuncdesc{
            Returns a list of validators, that will be called before the command execution.
            See \hyperref[sec:prerequisites_desc]{prerequisites description} for more information.\\
            Default: \python{[]}
        }

    \item \docfunc{verify_prerequisites(self, environment)}
        \docfuncdesc{
            Iterates over the list returned from the \hyperref[sec:prerequisites_command]{\python{prerequisites()}}
            method, calls each one of them with the environment as an argument, and raises the
            \python{PrerequisiteException} if they return \python{False}.
        }

    \item \docfunc{set_name(self, name)}
        \docfuncdesc{
            Allows the step to store the name that it was assigned at
            \hyperref[sec:add_steps]{\python{Checking.add_steps()}} call.
            It can then be used to better identify the outputs of the command execution (e.g. prefixing filenames).
        }

    \item \docfunc{get_configuration_status(self)}
        \docfuncdesc{
            Returns a 2-tuple containing the information if the command can be run, and an exit status if it cannot.
            Useful for the aggregate steps, where the command that needs to be run occurs to be undefined
            (e.g. because the file type passed as an argument is not recognized).\\
            Default: \python{(True, None)}
        }

    \item \phantomsection \label{sec:get_user} \docfunc{get_user(self)}
        \docfuncdesc{
            Returns the name of the user that the command should be run as.
        }

    \item \phantomsection \label{sec:get_group} \docfunc{get_group(self)}
        \docfuncdesc{
            Returns the name of the group that the command should be run as.
        }
\end{itemize}

\subsubsection{Pre-defined commands}\label{subsec:predefined_commands}

\subsubsection*{\python{CompileBase}}\label{subsec:CompileBase}

\begin{itemize}[label={}]
    \item Base class of every compilation command.
          Receives the compiler, files to be compiled, and compilation options as the constructor arguments, and prepares
          appropriate prerequisites (
          \hyperref[subsec:ProgramExistsPrerequisite]{\python{ProgramExistsPrerequisite}},
          \hyperref[subsec:FileExistsPrerequisite]{\python{FileExistsPrerequisite}},
          \hyperref[subsec:NonEmptyListPrerequisite]{\python{NonEmptyListPrerequisite}}
          ), postconditions (
          \hyperref[subsec:ExitCodePostcondition]{\python{ExitCodePostcondition}} $\rightarrow$ \code{CME}
          ), and a command to be called.
\end{itemize}

\subsubsection*{\python{CompileNasm}}\label{subsec:CompileNasm}

\begin{itemize}[label={}]
    \item Inherits from the \hyperref[subsec:CompileBase]{\python{CompileBase}} class, specifies the \code{nasm}
          compiler and \code{-felf64} as default compilation options.
\end{itemize}

\subsubsection*{\python{CompileC}}\label{subsec:CompileC}

\begin{itemize}[label={}]
    \item Inherits from the \hyperref[subsec:CompileBase]{\python{CompileBase}} class, specifies the \code{gcc}
          compiler.
\end{itemize}

\subsubsection*{\python{CompileCpp}}\label{subsec:CompileCpp}

\begin{itemize}[label={}]
    \item Inherits from the \hyperref[subsec:CompileBase]{\python{CompileBase}} class, specifies the \code{g++}
          compiler.
\end{itemize}

\subsubsection*{\python{CompileCSharp}}\label{subsec:CompileCSharp}

\begin{itemize}[label={}]
    \item Inherits from the \hyperref[subsec:CompileBase]{\python{CompileBase}} class, specifies the \code{mcs}
          compiler and \code{-t:exe -out:main.exe} as default compilation options.
\end{itemize}

\subsubsection*{\python{CompileGo}}\label{subsec:CompileGo}

\begin{itemize}[label={}]
    \item Inherits from the \hyperref[subsec:CompileBase]{\python{CompileBase}} class, specifies the \code{gccgo}
          compiler.
\end{itemize}

\subsubsection*{\python{CompileHaskell}}\label{subsec:CompileHaskell}

\begin{itemize}[label={}]
    \item Inherits from the \hyperref[subsec:CompileBase]{\python{CompileBase}} class, specifies the \code{ghc}
          compiler.
\end{itemize}

\subsubsection*{\python{CompileJava}}\label{subsec:CompileJava}

\begin{itemize}[label={}]
    \item Inherits from the \hyperref[subsec:CompileBase]{\python{CompileBase}} class, specifies the \code{javac}
          compiler.
\end{itemize}

\subsubsection*{\python{CreateJar}}\label{subsec:CreateJar}

\begin{itemize}[label={}]
    \item Takes the files that will be packed into a JAR archive, output file, manifest and an entrypoint as the
          constructor arguments.
          Prepares the standard prerequisites (
          \hyperref[subsec:ProgramExistsPrerequisite]{\python{ProgramExistsPrerequisite}},
          \hyperref[subsec:FileExistsPrerequisite]{\python{FileExistsPrerequisite}},
          \hyperref[subsec:NonEmptyListPrerequisite]{\python{NonEmptyListPrerequisite}}
          ), postconditions (
          \hyperref[subsec:ExitCodePostcondition]{\python{ExitCodePostcondition}} $\rightarrow$ \code{CME}
          ), and uses the \code{jar} program in the returned command.
\end{itemize}

\subsubsection*{\python{Link}}\label{subsec:Link}

\begin{itemize}[label={}]
    \item Takes the object files that will be linked together (using the \code{ld} program) and the output file as the
          constructor arguments.
          Specifies the standard prerequisites (
          \hyperref[subsec:ProgramExistsPrerequisite]{\python{ProgramExistsPrerequisite}},
          \hyperref[subsec:FileExistsPrerequisite]{\python{FileExistsPrerequisite}},
          \hyperref[subsec:NonEmptyListPrerequisite]{\python{NonEmptyListPrerequisite}}
          ), and postconditions (
          \hyperref[subsec:ExitCodePostcondition]{\python{ExitCodePostcondition}} $\rightarrow$ \code{CME}
          ).
\end{itemize}

\subsubsection*{\python{Make}}\label{subsec:Make}

\begin{itemize}[label={}]
    \item Takes the target and the build directory as the constructor arguments.
          Specifies only the \hyperref[subsec:ProgramExistsPrerequisite]{\python{ProgramExistsPrerequisite}}
          prerequisite, and sets the
          \hyperref[subsec:ExitCodePostcondition]{\python{ExitCodePostcondition}} $\rightarrow$ \code{CME}
          postcondition.
\end{itemize}

\subsubsection*{\python{CMake}}\label{subsec:CMake}

\begin{itemize}[label={}]
    \item Takes the source and the build directories as the constructor arguments.
          Specifies only the \hyperref[subsec:ProgramExistsPrerequisite]{\python{ProgramExistsPrerequisite}}
          prerequisite, and sets the
          \hyperref[subsec:ExitCodePostcondition]{\python{ExitCodePostcondition}} $\rightarrow$ \code{CME}
          postcondition.
\end{itemize}

\subsubsection*{\python{Run}}\label{subsec:Run}

\begin{itemize}[label={}]
    \item Base class for all of the commands that are running a program.
          Receives the executable, its command line arguments, and paths to the files representing the standard I/O
          streams.
          Sets the \hyperref[subsec:ProgramExistsPrerequisite]{\python{ProgramExistsPrerequisite}} on the
          executable, and \hyperref[subsec:FileExistsPrerequisite]{\python{FileExistsPrerequisite}} on the
          \shell{stdin}.
          No postconditions are set.
\end{itemize}

\subsubsection*{\python{RunSolution}}\label{subsec:RunSolution}

\begin{itemize}[label={}]
    \item Inherits from the \hyperref[subsec:Run]{\python{Run}} class and sets \code{./a.out} as the default
          executable.
          This class, and also all other following the \code{*Solution} convention define three postconditions:
          \hyperref[subsec:UsedTimePostcondition]{\python{UsedTimePostcondition}} $\rightarrow$ \code{TLE},
          \hyperref[subsec:UsedMemoryPostcondition]{\python{UsedMemoryPostcondition}} $\rightarrow$ \code{MEM}, and
          \hyperref[subsec:ExitCodePostcondition]{\python{ExitCodePostcondition}} $\rightarrow$ \code{RTE}.
\end{itemize}

\subsubsection*{\python{RunCSharp}}\label{subsec:RunCSharp}

\begin{itemize}[label={}]
    \item Inherits from the \hyperref[subsec:Run]{\python{Run}} class and sets \code{mono} as the default
          executable.
          Takes an EXE file to be run and the interpreter options as the arguments, and sets the
          \hyperref[subsec:FileExistsPrerequisite]{\python{FileExistsPrerequisite}} on that EXE file.
\end{itemize}

\subsubsection*{\python{RunPSQL}}\label{subsec:RunPSQL}

\begin{itemize}[label={}]
    \item Inherits from the \hyperref[subsec:Run]{\python{Run}} class and sets \code{psql} as the default
          executable.
          Receives a SQL file and the connection configuration: user, password, host and database name as the
          arguments.
\end{itemize}

\subsubsection*{\python{RunJavaClass}}\label{subsec:RunJavaClass}

\begin{itemize}[label={}]
    \item Inherits from the \hyperref[subsec:Run]{\python{Run}} class and sets \code{java} as the default
          executable.
          Takes a class file to be run and the interpreter options as the arguments, and sets the
          \hyperref[subsec:FileExistsPrerequisite]{\python{FileExistsPrerequisite}} on that class file.
\end{itemize}

\subsubsection*{\python{RunJar}}\label{subsec:RunJar}

\begin{itemize}[label={}]
    \item Inherits from the \hyperref[subsec:Run]{\python{Run}} class and sets \code{java} as the default
          executable.
          Takes a JAR file to be run and the interpreter options as the arguments, and sets the
          \hyperref[subsec:FileExistsPrerequisite]{\python{FileExistsPrerequisite}} on that JAR file.
\end{itemize}

\subsubsection*{\python{RunPython}}\label{subsec:RunPython}

\begin{itemize}[label={}]
    \item Inherits from the \hyperref[subsec:Run]{\python{Run}} class and sets \code{python} as the default
          executable.
          Takes a Python script to be run and the interpreter options as the arguments, and sets the
          \hyperref[subsec:FileExistsPrerequisite]{\python{FileExistsPrerequisite}} on that Python script.
\end{itemize}

\subsubsection*{\python{RunShell}}\label{subsec:RunShell}

\begin{itemize}[label={}]
    \item This class is synonymous to the \hyperref[subsec:Run]{\python{Run}} class.
\end{itemize}

\subsubsection*{\python{Diff}}\label{subsec:Diff}

\begin{itemize}[label={}]
    \item Command to compare two files while ignoring trailing spaces.
          Takes these two files as the arguments, sets the
          \hyperref[subsec:FileExistsPrerequisite]{\python{FileExistsPrerequisite}} on them, and defines the
          \hyperref[subsec:ExitCodePostcondition]{\python{ExitCodePostcondition}} $\rightarrow$ \code{ANS}
          postcondition.
\end{itemize}

\subsubsection*{\python{RunChecker}}\label{subsec:RunChecker}

\begin{itemize}[label={}]
    \item Inherits from the \hyperref[subsec:Run]{\python{Run}} class and sets the
          \hyperref[subsec:ExitCodePostcondition]{\python{ExitCodePostcondition}} $\rightarrow$ \code{ANS}
          postcondition.
\end{itemize}

\subsubsection*{\python{ExtractArchive}}\label{subsec:ExtractArchive}

\begin{itemize}[label={}]
    \item Command to extract various types of archives.
          Receives the archive, optionally its type and a directory to extract to as the arguments.
          Declares the \hyperref[subsec:ProgramExistsPrerequisite]{\python{ProgramExistsPrerequisite}} on the
          appropriate executable (e.g. \code{unzip} or \code{tar}), and the
          \hyperref[subsec:ExitCodePostcondition]{\python{ExitCodePostcondition}} $\rightarrow$ \code{EXT}
          postcondition.
          Return the \code{EXT} exit status if the archive type is not recognized.
\end{itemize}
