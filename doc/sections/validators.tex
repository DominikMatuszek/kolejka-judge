\section{Validators}\label{sec:validators}

    Validators provide a way to verify that the state of an environment is as expected.
    There are two types of validators: those executed before a step (\textit{prerequisites}), and those executed after
    (\textit{postconditions}). \\

    \phantomsection \label{sec:prerequisites_desc} Prerequisites are called with an environment as the only argument.
    They return \python{True} when they find the state of this environment acceptable, and \python{False} otherwise.
    If any of the prerequisites is not met, \python{PrerequisiteException} is raised from the
    \python{verify_prerequisites()} method, and the checking terminates.
    Implementation of a prerequisite can be either a function, or a class implementing the \python{__call__()} method.
    Classes are preferred because of their ability to be passed arguments in their constructors, and the possibility of
    overriding the \python{__repr__()} behaviour, resulting in a better exception message in case the requirement is not
    satisfied.\\

    \phantomsection \label{sec:postconditions_desc} Postconditions are given the step's execution statistics as their
    argument, and return \python{True} if they find them acceptable, and \python{False} otherwise.
    When first from the list of assigned to the step postconditions (consisting of the 2-tuples in the form
    \python{(postcondition, exit_status)}) fails to meet its criteria, \python{exit_status} is returned from the
    \python{verify_postconditions()} method, and consequently as the final step's
    exit status.
    This in turn will cause the checking to halt immediately, and no following steps will be run.
    Similarly to the prerequisites, their implementation can be either a function, or a class implementing the
    \python{__call__()} method - with the same advantages of the class approach as previously, except of the
    \python{__repr__()} method, which is not used, due to the lack of raised exceptions.\\

    The following validators are implemented:

\subsection*{\python{FileExistsPrerequisite(file)}}\label{subsec:FileExistsPrerequisite}
\begin{itemize}[label={}]
    \item Calls the \python{file_exists()} validator from
          \hyperref[sec:Validators]{\python{ExecutionEnvironment.Validators}}.
\end{itemize}

\subsection*{\python{ProgramExistsPrerequisite(file)}}\label{subsec:ProgramExistsPrerequisite}
\begin{itemize}[label={}]
    \item Calls the \python{program_exists()} validator from
          \hyperref[sec:Validators]{\python{ExecutionEnvironment.Validators}}.
\end{itemize}

\subsection*{\code{NonEmptyListPrerequisite(list)}}\label{subsec:NonEmptyListPrerequisite}
\begin{itemize}[label={}]
    \item Makes sure that the given list is not empty.
          This is especially useful when dealing with source files using glob patterns, which tend to expand to an empty
          list in case of a mistake - even when there are no wildcard characters.
          Adopting this validator can result in an earlier detection of the problem.
\end{itemize}

\subsection*{\code{ExitCodePostcondition(allowed_codes)}}\label{subsec:ExitCodePostcondition}
\begin{itemize}[label={}]
    \item Checks whether the command have terminated with a code that is present in the \python{allowed_codes}.
          The argument defaults to \python{[0]}, i.e. all non-zero exit codes are forbidden.
\end{itemize}

\subsection*{\code{UsedTimePostcondition(time)}}\label{subsec:UsedTimePostcondition}
\begin{itemize}[label={}]
    \item Checks whether the command took less than \python{time} seconds to execute.
\end{itemize}

\subsection*{\code{UsedMemoryPostcondition(memory)}}\label{subsec:UsedMemoryPostcondition}
\begin{itemize}[label={}]
    \item Checks whether the command allocated less than \python{memory} bytes during the execution.
\end{itemize}

\subsection*{\code{PSQLErrorPostcondition()}}\label{subsec:PSQLErrorPostcondition}
\begin{itemize}[label={}]
    \item Opens the \shell{stderr} file from execution statistics, if it exists, and ensures that the
          \python{'^.*ERROR:'} expression is not matched.
          Implementation was based on one of the existing judges.
\end{itemize}
