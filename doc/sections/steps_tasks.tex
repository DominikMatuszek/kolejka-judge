\subsection{Tasks}\label{subsec:steps_tasks}
    The base class of every task step is \python{TaskBase}.

\subsection*{\python{TaskBase}}\label{subsec:TaskBase}

\begin{itemize}[label={}]
    \item \phantomsection \label{sec:execute} \docfunc{execute(self, environment)}
        \docfuncdesc{
            Represents the Python code part of the checking pipeline.
            It will be called by the
            \hyperref[sec:run_task_step]{\python{ExecutionEnvironment.run_task_step()}} method.
            Returns a 2-tuple: \python{(exit_status, execution_statistics)}.
            Both fields can be \python{None}.
        }

    \item \docfunc{set_name(self, name)}
        \docfuncdesc{
            Allows the step to store the name that it was assigned at
            \hyperref[sec:add_steps]{\python{Checking.add_steps()}} call.
            It can then be used to better identify the outputs of the command execution (e.g. prefixing filenames).
        }

    \item \phantomsection \label{sec:prerequisites_task} \docfunc{prerequisites(self)}
        \docfuncdesc{
            Returns a list of validators, that will be called before the command execution.
            See \hyperref[sec:prerequisites_desc]{prerequisites description} for more information.\\
            Default: \python{[]}
        }

    \item \docfunc{verify_prerequisites(self, environment)}
        \docfuncdesc{
            Iterates over the list returned from the \hyperref[sec:prerequisites_task]{\python{prerequisites()}} method,
            calls each one of them with the environment as an argument, and raises the \python{PrerequisiteException}
            if they return \python{False}.
        }
\end{itemize}

\subsubsection{Pre-defined tasks}\label{subsec:predefined_tasks}

\subsubsection*{\python{AutoCompile}}\label{subsec:AutoCompile}

\begin{itemize}[label={}]
    \item Detects the compiler and compiles the file passed as an argument, based on its extension.
          Creates an \code{autocompile/run.sh} file that contains the shell command that can be then run.
          Useful when creating custom checkers - code written in one language can then be quickly replaced with another
          written in second language, with no or only minimal changes in the judge script itself.
          The prerequisites set are the same as of the detected command that will be used to compile the file
          internally.
\end{itemize}

\subsubsection*{\python{RenameJavaFile}}\label{subsec:RenameJavaFile}

\begin{itemize}[label={}]
    \item Renames the Java class file given as the argument, to correspond to the public class and package contained in
          that file.
          Implementation was based on one of the existing judges.
\end{itemize}

\subsubsection*{\python{ListFiles}}\label{subsec:ListFiles}

\begin{itemize}[label={}]
    \item Takes glob patterns and a environment's variable name (from where it can be retrieved using
          \hyperref[sec:DependentExpr]{\python{DependentExpr}}) as the arguments.
          Evaluates the glob expressions and stores the result using
          \hyperref[sec:set_variable]{\python{ExecutionEnvironment.set_variable()}} method.
\end{itemize}

\subsubsection*{\python{CreateUser}}\label{subsec:CreateUser}

\begin{itemize}[label={}]
    \item Creates the user with name received as an argument, using the \shell{useradd} program.
\end{itemize}

\subsubsection*{\python{Setuid}}\label{subsec:Setuid}

\begin{itemize}[label={}]
    \item Receives a file as an argument and sets the \shell{setuid} bit on it.
\end{itemize}
